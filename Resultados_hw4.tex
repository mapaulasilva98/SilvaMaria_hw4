\documentclass{article}

\title{Tearea 4}
\author{Maria Paula Silva Wilches }
\date{Noviembre 2018}

\usepackage{graphicx}
\usepackage[utf8]{inputenc}

\begin{document}

\maketitle

\section{ODE}

\begin{figure}[htb]
	\centering
	\includegraphics[scale=0.5]{datos45.png}
	\caption{Trayectoria para ángulo de 45 grados.}
\end{figure}



\begin{figure}[htb]
	\centering
	\includegraphics[scale=0.5]{datosangulos.png}
	\caption{Trayectoria para ángulo de 45 grados.}
\end{figure}


\section{PDE}

\subsection{CASO 1}

\begin{figure}[htb]
	\centering
	\includegraphics[scale=0.5]{datosmatrizinicial1.png}
	\caption{grafica de las condiciones iniciales de mi sistema para el caso 1.}
\end{figure}

\begin{figure}[htb]
	\centering
	\includegraphics[scale=0.5]{datospromedio1.png}
	\caption{grafica las temperaturas promedio en funcion del timepo para el caso 1.}
\end{figure}

\begin{figure}[htb]
	\centering
	\includegraphics[scale=0.5]{datosinter1_1.png}
	\caption{grafica de los datos internos del sistema 1-1.}
\end{figure}

\begin{figure}[htb]
	\centering
	\includegraphics[scale=0.5]{datosinter1_2.png}
	\caption{grafica de los datos internos del sistema 1-2.}
\end{figure}


\begin{figure}[htb]
	\centering
	\includegraphics[scale=0.5]{datosequilibrio1.png}
	\caption{Distribución de temperatura en el estado de equilibrio del sistema para el caso 1.}
\end{figure}







\begin{figure}[htb]
	\centering
	\includegraphics[scale=0.5]{datosmatrizinicial2.png}
	\caption{grafica de las condiciones iniciales de mi sistema para el caso 2.}
\end{figure}

\begin{figure}[htb]
	\centering
	\includegraphics[scale=0.5]{datospromedio2.png}
	\caption{grafica las temperaturas promedio en funcion del timepo para mi caso 2.}
\end{figure}

\begin{figure}[htb]
	\centering
	\includegraphics[scale=0.5]{datosinter2_1.png}
	\caption{grafica de los datos internos del sistema 2-1.}
\end{figure}

\begin{figure}[htb]
	\centering
	\includegraphics[scale=0.5]{datosinter2_2.png}
	\caption{grafica de los datos internos del sistema 2-2.}
\end{figure}


\begin{figure}[htb]
	\centering
	\includegraphics[scale=0.5]{datosequilibrio2.png}
	\caption{Distribución de temperatura en el estado de equilibrio del sistema para el caso 2.}
\end{figure}





\begin{figure}[htb]
	\centering
	\includegraphics[scale=0.5]{datosmatrizinicial3.png}
	\caption{grafica de las condiciones iniciales de mi sistema para el caso 3.}
\end{figure}

\begin{figure}[htb]
	\centering
	\includegraphics[scale=0.5]{datospromedio3.png}
	\caption{grafica las temperaturas promedio en funcion del timepo para mi caso 3.}
\end{figure}

\begin{figure}[htb]
	\centering
	\includegraphics[scale=0.5]{datosinter3_1.png}
	\caption{grafica de los datos internos del sistema 3-1.}
\end{figure}

\begin{figure}[htb]
	\centering
	\includegraphics[scale=0.5]{datosinter3_2.png}
	\caption{grafica de los datos internos del sistema 3-2.}
\end{figure}


\begin{figure}[htb]
	\centering
	\includegraphics[scale=0.5]{datosequilibrio3.png}
	\caption{Distribución de temperatura en el estado de equilibrio del sistema para el caso 3.}
\end{figure}



\end{document}